
\section{Informe sobre la Clase \texttt{TableroGUI}}

La clase \texttt{TableroGUI} se encarga de crear una interfaz gráfica para visualizar el tablero de soldados en el juego. A continuación, se presenta un análisis detallado de su estructura y funcionalidades, así como ejemplos de cómo interactúa con otras clases.

\subsection{Características Principales}

\subsubsection{Herencia de \texttt{JFrame}}
\begin{lstlisting}[language=Java]
public class TableroGUI extends JFrame
\end{lstlisting}
La clase \texttt{TableroGUI} hereda de \texttt{JFrame}, la cual es una clase que proporciona una ventana contenedora para la interfaz gráfica.

\subsubsection{Atributos}
\begin{lstlisting}[language=Java]
private Soldado[][] tablero;
\end{lstlisting}
\texttt{tablero}: Almacena la información del tablero de soldados que se mostrará en la interfaz.

\subsubsection{Constructor}
\begin{lstlisting}[language=Java]
public TableroGUI(Soldado[][] tablero)
\end{lstlisting}
Este constructor recibe una matriz de soldados (\texttt{tablero}) como parámetro y configura la interfaz. Establece el título, tamaño de la ventana, operación de cierre y visibilidad. Crea un objeto \texttt{TableroPanel} (clase interna) y lo agrega a la ventana.

\subsubsection{Clase Interna \texttt{TableroPanel}}
\begin{lstlisting}[language=Java]
private class TableroPanel extends JPanel
\end{lstlisting}
Esta clase interna hereda de \texttt{JPanel} y se encarga de dibujar el tablero en la interfaz.

\subsubsection{Método \texttt{paintComponent}}
\begin{lstlisting}[language=Java]
@Override
protected void paintComponent(Graphics g)
\end{lstlisting}
Este método sobrescrito se encarga de dibujar los elementos en el panel. Itera sobre la matriz de soldados y pinta las celdas del tablero. Utiliza colores diferentes para representar a los dos ejércitos.

\subsubsection{Ejemplo de Uso en la Clase \texttt{Videojuego}}
\begin{lstlisting}[language=Java]
SwingUtilities.invokeLater(() -> {
    new TableroGUI(tablero);
});
\end{lstlisting}
Cuando se crea un objeto \texttt{TableroGUI} en la clase \texttt{Videojuego}, se pasa la matriz de soldados (\texttt{tablero}) como parámetro. Esto muestra la interfaz gráfica del tablero.

\subsection{Interacción con Otras Clases}

\subsubsection{Uso del Tablero en \texttt{Videojuego}}
\begin{lstlisting}[language=Java]
SwingUtilities.invokeLater(() -> {
    new TableroGUI(tablero);
});
\end{lstlisting}
La clase \texttt{Videojuego} utiliza \texttt{TableroGUI} para mostrar gráficamente el estado del tablero de soldados en el juego. La llamada a \texttt{SwingUtilities.invokeLater} asegura que la interfaz se actualice de manera segura.

\subsubsection{Acceso a la Información de los Soldados}
\begin{lstlisting}[language=Java]
Soldado soldado = tablero[fila][columna];
\end{lstlisting}
Dentro del método \texttt{paintComponent}, se accede a la información de cada soldado en el tablero para determinar su posición y tipo. Esto permite representar visualmente la ubicación y afiliación de cada soldado en el tablero.

\subsubsection{Ejemplo de Colores según Afiliación}
\begin{lstlisting}[language=Java]
if ((nombre.substring(nombre.length()-2, nombre.length()-1).equals("Z"))) {
    g.setColor(Color.CYAN);
} else {
    g.setColor(Color.ORANGE);
}
\end{lstlisting}
El código verifica el símbolo del soldado para determinar a qué ejército pertenece y utiliza colores diferentes (cian u naranja) para representarlos en la interfaz.
