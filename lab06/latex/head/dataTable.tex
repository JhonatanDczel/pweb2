
	
	\vspace*{10pt}
	
	\begin{center}	
		\fontsize{17}{17} \textbf{ Informe de Laboratorio \itemPracticeNumber}
	\end{center}
	\centerline{\textbf{\Large Tema: \itemTheme}}
	\vspace*{0.5cm}	

	\begin{flushright}
		\begin{tabular}{|M{2.5cm}|N|}
			\hline 
			\rowcolor{tablebackground}
			\color{white} \textbf{Nota}  \\
			\hline 
			     \\[30pt]
			\hline 			
		\end{tabular}
	\end{flushright}	

	\begin{table}[H]
		\begin{tabular}{|x{4.9cm}|x{4.3cm}|x{5.1cm}|}
			\hline 
			\rowcolor{tablebackground}
			\color{white} \textbf{Estudiante} & \color{white}\textbf{Escuela}  & \color{white}\textbf{Asignatura}   \\
			\hline 
			{\itemStudent \par \itemEmail} & \itemSchool & {\itemCourse \par Semestre: \itemSemester \par Código: \itemCourseCode}     \\
			\hline 			
		\end{tabular}
	\end{table}		
	
	\begin{table}[H]
		\begin{tabular}{|x{4.7cm}|x{4.8cm}|x{4.8cm}|}
			\hline 
			\rowcolor{tablebackground}
			\color{white}\textbf{Laboratorio} & \color{white}\textbf{Tema}  & \color{white}\textbf{Duración}   \\
			\hline 
			\itemPracticeNumber & \itemTheme & 04 horas   \\
			\hline 
		\end{tabular}
	\end{table}
	
	\begin{table}[H]
		\begin{tabular}{|x{4.7cm}|x{4.8cm}|x{4.8cm}|}
			\hline 
			\rowcolor{tablebackground}
			\color{white}\textbf{Semestre académico} & \color{white}\textbf{Fecha de inicio}  & \color{white}\textbf{Fecha de entrega}   \\
			\hline 
			\itemAcademic & \itemInput &  \itemOutput  \\
			\hline 
		\end{tabular}
	\end{table}


\section{Actividades}
\begin{itemize}
	\item Elabore un primer informe grupal, con el modelo de datos de una aplicación que desarrollará durante este semestre.
	\item Utilicen todas las recomendaciones encontradas en la aplicación library.
\end{itemize}

\subsection{Pregunta}
Por cada integrante del equipo, resalte un aprendizaje que adquirió al momento de estudiar esta primera parte de Django (Admin). No se reprima de ser detallista. Coloque su nombre entre parentesis para saber que es su aporte.

	\section{Equipos, materiales y temas utilizados}
	\begin{itemize}
		\item Sistema Operativo ArchCraft GNU Linux 64 bits Kernell
		\item Termux
		\item NeoVim
		\item Git 2.42.0
		\item Cuenta en GitHub con el correo institucional.
    \item Latex
    \item Python 3.10.2
		\item Django 4.0.2
		\item Pip
		\item Virtualenv
		\item Figma
		\item Graphviz
	\end{itemize}

	\section{URL de Repositorio Github}
	\begin{itemize}
            \item URL del Repositorio GitHub para clonar o recuperar.
            \item \url{https://github.com/JhonatanDczel/CoderDojo-UNSA.git}
	\end{itemize}
