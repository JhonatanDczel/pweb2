
	
	\vspace*{10pt}
	
	\begin{center}	
		\fontsize{17}{17} \textbf{ Informe de Laboratorio \itemPracticeNumber}
	\end{center}
	\centerline{\textbf{\Large Tema: \itemTheme}}
	\vspace*{0.5cm}	

	\begin{flushright}
		\begin{tabular}{|M{2.5cm}|N|}
			\hline 
			\rowcolor{tablebackground}
			\color{white} \textbf{Nota}  \\
			\hline 
			     \\[30pt]
			\hline 			
		\end{tabular}
	\end{flushright}	

	\begin{table}[H]
		\begin{tabular}{|x{4.9cm}|x{4.3cm}|x{5.1cm}|}
			\hline 
			\rowcolor{tablebackground}
			\color{white} \textbf{Estudiante} & \color{white}\textbf{Escuela}  & \color{white}\textbf{Asignatura}   \\
			\hline 
			{\itemStudent \par \itemEmail} & \itemSchool & {\itemCourse \par Semestre: \itemSemester \par Código: \itemCourseCode}     \\
			\hline 			
		\end{tabular}
	\end{table}		
	
	\begin{table}[H]
		\begin{tabular}{|x{4.7cm}|x{4.8cm}|x{4.8cm}|}
			\hline 
			\rowcolor{tablebackground}
			\color{white}\textbf{Laboratorio} & \color{white}\textbf{Tema}  & \color{white}\textbf{Duración}   \\
			\hline 
			\itemPracticeNumber & \itemTheme & 04 horas   \\
			\hline 
		\end{tabular}
	\end{table}
	
	\begin{table}[H]
		\begin{tabular}{|x{4.7cm}|x{4.8cm}|x{4.8cm}|}
			\hline 
			\rowcolor{tablebackground}
			\color{white}\textbf{Semestre académico} & \color{white}\textbf{Fecha de inicio}  & \color{white}\textbf{Fecha de entrega}   \\
			\hline 
			\itemAcademic & \itemInput &  \itemOutput  \\
			\hline 
		\end{tabular}
	\end{table}


\section{Actividades}
Elabore un informe individual, de estudio de Git y GitHub.

Cree su página personal con las recomendaciones dadas por la W3C en su tutorial de CSS: Empezando con HTML y CSS: \url{https://www.w3.org/Style/Examples/011/firstcss.en.html}.

\begin{itemize}
    \item \textbf{index.html} - (Menú principal - Página principal de bienvenida al sitio beb)
    
    \item \textbf{autor.html} - (Menú principal - Página de presentación del autor) (Se activa Menú Izquierdo)
    
    \item \textbf{hobbies.html} - (Menú Izquierdo - Página de fotos y descripciones de sus hobbies.)
    
    \item \textbf{ingSoftware.html} - (Menú Izquierdo - Página donde explique ¿Qué es la Ing. de Software desde su punto de vista?. Descripción, especialidades que resalte.)
    
    \item \textbf{galeria.html} - (Menú Izquierdo - Página de fotos y descripciones libres que quiera compartir.)
    
    \item \textbf{estandaresWeb.html} - (Menú principal - Página donde describa la W3C y algunos estándares web.)
    
    \item \textbf{contactame.html} - (Menú principal - Página donde muestra un formulario de contacto.)
\end{itemize}

\textbf{Menú Principal:} |Inicio|Autor|Estándares Web|Contáctame|

\begin{itemize}
    \item \textbf{Inicio:} Página de presentación del sitio. "Bienvenido a ...".
    
    \item \textbf{Autor:} Descripción del autor. "Mi nombre es ...".
    
    \item \textbf{Estándares Web:} SVG, WOFF, WebRTC, XML.
    
    \item \textbf{Contáctame:} Formulario: |nombres:input| |correo:email| |genero:radiobutton| |nacimiento:date| |asunto:input| |contenido:textarea| |enviar:button|
\end{itemize}

\textbf{Menú Izquierdo:} |Autor| -> |Hobbies|Ing. de Software|Galería|

\begin{itemize}
    \item \textbf{Hobbies:} Descripción, tiempo, recomendaciones, fotos.
    
    \item \textbf{Ing. de Software:} Acerca de la carrera de Ing. de Software. Futuro. Especialidad.
    
    \item \textbf{Galería:} Fotos en barrio, universidad, trabajo u otras actividades, mascota.
\end{itemize}

\textbf{Contactanos:} Formulario que guarda en una base de datos. Las consutas que se desean enviar. Nombres, correo electrónico, asunto, detalle.

Utilice todas las recomendaciones dadas por el docente:

\begin{itemize}
    \item \textbf{Antecedentes:} Describir antecedentes previos que sean necesarios para desarrollar el laboratorio. (Las entregadas por el docente y/o las que se buscaron personalmente).
    
    \item \textbf{No usar JavaScript.} Sólo HTML y CSS.
    
    \item \textbf{Commits:} Elaborar la lista de envíos que permitiran culminar el laboratorio. (Previo a la implementación)
    
    \item \textbf{Source:} Explicar porciones de código fuente importantes, trascendentales que permitieron resolver el laboratorio y que reflejen su particularidad única. (Sólo en trabajos grupales se permite duplicidad)
    
    \item \textbf{Ejecución:} Muestra comandos, capturas de pantalla, explicando la forma de replicar y ejecutar el entregable del laboratorio.
\end{itemize}

\textbf{Validación:} Muestra validación HTML y CSS.

Utilice DockerFile para realizar operaciones automatizadas en Docker (incluido arrancar el servidor web a través de un puerto y copiar el proyecto web para acceder desde la máquina anfitrión.)

Ejemplo: \url{http://127.0.0.1:8085/lab02/}

\textbf{M.Sc. Ing. Richart Smith Escobedo Quispe Pág. 13 Programación Web}

\textbf{Universidad Nacional de San Agustín de Arequipa}

\textbf{Facultad de Ingeniería de Producción y Servicios}

\textbf{Departamento de Ingeniería de Sistemas e Informática}

\textbf{Escuela Profesional de Ingeniería de Sistemas}

\textbf{Programación Web}
		
	\section{Equipos, materiales y temas utilizados}
	\begin{itemize}
		\item Sistema Operativo ArchCraft GNU Linux 64 bits Kernell
		\item NeoVim
		\item Git 2.42.0
		\item Cuenta en GitHub con el correo institucional.
    \item HTML5
    \item CSS3
    \item Latex
	\end{itemize}
	\section{URL de Repositorio Github}
	\begin{itemize}
            \item URL del Repositorio GitHub para clonar o recuperar.
            \item \url{https://github.com/JhonatanDczel/pweb2.git}
            \item URL para el laboratorio \itemPracticeNumber{} en el Repositorio GitHub.
            \item \itemUrl
	\end{itemize}
