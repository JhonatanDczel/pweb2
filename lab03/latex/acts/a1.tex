\subsection{Página Personal}

La página personal que se desarrollará en el laboratorio 02 sigue los estándares web actuales e incluye las siguientes secciones:

\begin{itemize}
    \item \textbf{Autor}
    \item \textbf{Contáctame}
    \item \textbf{Estándares web}
    \item \textbf{Inicio}
    \item \textbf{Galería}
\end{itemize}

\subsubsection{Mejoras propuestas:}

\begin{itemize}
    \item \textbf{Autor:} Agregar una breve biografía del autor, destacando su experiencia, habilidades y proyectos relevantes.
    \item \textbf{Contáctame:} Implementar un formulario de contacto para facilitar la comunicación con los visitantes.
    \item \textbf{Estándares web:} Ofrecer información sobre los estándares web utilizados en el desarrollo de la página, promoviendo prácticas de codificación eficientes y accesibles.
    \item \textbf{Inicio:} Brindar una introducción atractiva y concisa sobre el propósito de la página personal, invitando a los usuarios a explorar su contenido.
    \item \textbf{Galería:} Enlazar la sección de galería con una colección visualmente atractiva de proyectos destacados del autor.
\end{itemize}

\subsubsection{Responsive Design:}

El archivo \texttt{responsive.css} contiene reglas de estilo que garantizan que la página se adapte correctamente a diferentes tamaños de pantalla. Destacan las siguientes características:

\begin{verbatim}
@media screen and (max-width: 850px) {
  nav {
    justify-content: space-evenly;
  }
}
\end{verbatim}

\begin{verbatim}
@media screen and (max-width: 460px) {
  html {
    font-size: 14px;
  }
  nav > .links > button {
    display: none;
  }
  main > .presentation-container > img {
    height: 250px;
  }
}
\end{verbatim}

\begin{verbatim}
@media screen and (max-width: 610px) {
  .project-card {
    width: 70vw;
  }
  .project-image {
    height: calc(0.72 * 70vw);
  }

  .projects-container {
    display: flex;
    flex-wrap: wrap;
    justify-content: center;
  }
}
\end{verbatim}

\subsubsection{Mejoras sugeridas:}

\begin{itemize}
    \item \textbf{Optimización de imágenes:} Implementar técnicas de compresión de imágenes para reducir el tiempo de carga en dispositivos móviles y mejorar la experiencia del usuario.
    \item \textbf{Pruebas exhaustivas:} Realizar pruebas en una variedad de dispositivos y navegadores para garantizar que la página se vea y funcione correctamente en diferentes entornos.
\end{itemize}

\subsubsection{Galería:}

La estructura de la galería hace uso de tecnologías como Grid y Flexbox para maquetar la presentación de proyectos de manera eficiente. Destacan las siguientes características:

\begin{verbatim}
.presentation-container {
  display: flex;
  flex-direction: column;
  gap: 3rem;
}
\end{verbatim}

\begin{verbatim}
.projects-container {
  width: 100%;
  display: grid;
  grid-template-columns: repeat(auto-fit, minmax(430px, 1fr));
  align-items: start;
  justify-items: center;
  grid-row-gap: 4rem;
  grid-column-gap: 2rem;
}
\end{verbatim}
